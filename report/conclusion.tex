\newpage
\section*{Заключение}
\addcontentsline{toc}{section}{Заключение}
В~результате проделанной работы авторы смогли решить все из подзадач, составляющих проблему автоматической парковки мобильного робота.
Ими успешно:
\begin{itemize}
    \item была собрана машинка с рулевым управлением и составлена ее математическая модель;
    \item были решены вопросы, касающиеся управления ее движением и локализации (локальной ориентации): в первом из случаев была построена система управления, дающая роботу возможность двигаться по желаемой траектории, а во втором~--- подобраны датчики и выбраны алгоритмы расчета, позволяющие получать хорошую оценку положения робота в пространстве, что, в свою очередь, было проверено с помощью сторонней системы технического зрения;
    \item был разработан метод (алгоритм) картирования окрестностей предполагаемого места парковки и поиска последнего;
    \item был синтезирован алгоритм планирования траектории парковочного движения.
\end{itemize}
При этом, однако, из-за нехватки (или нерационального использования) времени система целиком протестирована не была.

Исходный код данного отчета и относящихся к проекту программ (рассчетных, управляющих и проч.) доступны по адресу \verb|https://github.com/kirillin/parking-lego-car|.