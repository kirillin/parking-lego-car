\newpage
\section{Поиск парковочного места}

В этой работе рассматривается правостороннее движение, поэтому парковка производится с правой стороны дороги. 

Поиск парковочного места осуществляется по тому же принципу, что и в реальной жизни: робот-машинка двигается параллельно припаркованному ряду автомобилей с постоянной скоростью $v$, одновременно сканируя пространство справа от себя. Сканирование олицетворяет собой измерение расстояния между роботом и объектами справа от него.
Измерения производятся двумя ультразвуковыми дальномерами ($S_{front}$ и $S_{rear}$), расположение которых на роботе изображено на рисунке~\ref{robot_and_us_sensors}.

% Важно! Передний датчик покрасить в красный, задний в синий!
\begin{figure}[h!]
	\begin{minipage}[h]{0.47\textwidth}
		\centering{ \includegraphics[width=\textwidth, draft]{first.jpg} \\ a) вид сверху}
	\end{minipage}
	\hfill
	\begin{minipage}[h]{0.47\textwidth}
		\centering{ \includegraphics[width=\textwidth, draft]{second.jpg} \\ б) вид справа (по ходу движения)}
	\end{minipage}
	\caption{Расположение ультразвуковых дальномеров.}
	\label{robot_and_us_sensors}
\end{figure}

Измерения показаний ультразвуковых дальномеров представляется в локальной СК робота в точками ($p_{1}$ и $p_{2}$)~\eqref{us_points}. После чего преобразуются в глобальную СК путем перемножения матрицы однородных преобразования $H$ и соответствующей измеренной точки~\eqref{}--\eqref{label}

\begin{equation}\label{us_points}
p_{1} = 
\begin{bmatrix}
x_{p_1} \\
y_{p_1} \\
\end{bmatrix}
=
\begin{bmatrix}
L \\
- (S_{front} + d) \\
\end{bmatrix}\!\!;~
p_{2} = 
\begin{bmatrix}
x_{p_2} \\
y_{p_2} \\
\end{bmatrix}
=
\begin{bmatrix}
-l \\
- (S_{rear} + d) \\
\end{bmatrix}\!\!.
\end{equation}
где $L$~--- база робота-машинки; $l$~--- расстояние между задней осью и ультразвуковым дальномером, закрепленным на задней части робота; $d$~--- расстояние от продольной оси робота до сенсоров дальномеров.

\begin{equation}
	H = 
	\begin{bmatrix}
		\cos{\theta} & -\sin{\theta} & 0 & x \\
		\sin{\theta} & \cos{\theta} & 0 & y \\
		0 & 0 & 1 & 0\\
		0 & 0 & 0 & 1
	\end{bmatrix}
\end{equation}

\begin{equation}
	P = H * p = 
	\begin{bmatrix}
		x_P\\ y_P\\0\\1
	\end{bmatrix}
	= 
	\begin{bmatrix}
		\cos{\theta} & -\sin{\theta} & 0 & x \\
		\sin{\theta} & \cos{\theta} & 0 & y \\
		0 & 0 & 1 & 0\\
		0 & 0 & 0 & 1
	\end{bmatrix} 
	*
	\begin{bmatrix}
		x_p \\ y_p \\ 0\\ 1
	\end{bmatrix}
\end{equation}

В результате описанного выше сканирования строится карта, на которую наносятся проекции препятствий на плоскость дорожного полотна. На рисунке~\ref{map} показан пример такой карты.

\begin{figure}[h!]
	\begin{minipage}[h]{0.47\textwidth}
		\centering{ \includegraphics[width=\textwidth, draft]{first.jpg} \\ a) реальная карта}
	\end{minipage}
	\hfill
	\begin{minipage}[h]{0.47\textwidth}
		\centering{ \includegraphics[width=\textwidth, draft]{second.jpg} \\ б) результаты сканирования}
	\end{minipage}
	\caption{Пример карты.}
	\label{map}
\end{figure}


